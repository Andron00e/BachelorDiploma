\newpage


\begin{thebibliography}{99}
	\bibitem{kwapisz2010}
	\textit{J. R. Kwapisz, G. M. Weiss, S. A. Moore} Activity Recognition using Cell Phone Accelerometers~// Proceedings of the Fourth International Workshop on Knowledge Discovery from Sensor Data, 2010. Vol. 12. P. 74--82.
	
	\bibitem{wang2014}
	\textit{W. Wang, H. Liu, L. Yu, F. Sun} Activity Recognition using Cell Phone Accelerometers~// Joint Conference on Neural Networks, 2014. P. 1185--1190.
	
	\bibitem{Ignatov2015}
	\textit{A. D. Ignatov, V. V. Strijov} Human activity recognition using quasiperiodic time series collected from a single tri-axial accelerometer.~// Multimedial Tools and Applications, 2015.
	
	\bibitem{Olivares2012}
	\textit{A. Olivares, J. Ramirez, J. M. Gorris, G. Olivares, M. Damas} Detection of (in)activity periods in human body motion using inertial sensors: A comparative study.~// Sensors, 12(5):5791–5814, 2012.
	
	\bibitem{cinar2018}
	\textit{Y. G. Cinar and H. Mirisaee} Period-aware content attention RNNs for time series forecasting with missing values~// Neurocomputing, 2018. Vol. 312. P. 177--186.
	
	\bibitem{motrenko2015}
	\textit{A. P. Motrenko, V. V. Strijov} Extracting fundamental periods to segment biomedical signals~// Journal of Biomedical and Health Informatics, 2015,~20(6). P.~1466~-~1476.
	
	\bibitem{lukashin2003}
	\textit{Y. P. Lukashin} Adaptive methods for short-term forecasting~// Finansy and Statistik, 2003.
	
	\bibitem{Ivkin2015}
	\textit{И. П. Ивкин,  М. П. Кузнецов} Алгоритм классификации временных рядов акселерометра по комбинированному признаковому описанию.~// Машинное обучение и анализ данных, 2015.
	
	\bibitem{Katrutsa2015}
	\textit{V. V. Strijov, A. M. Katrutsa} Stresstes procedures for features selection algorithms.~// Schemometrics and Intelligent Laboratory System, 2015.
	
	\bibitem{Borg2005}
	\textit{I. Borg, P. J. F. Groenen} Modern Multidimensional Scaling. --- New York: Springer, 2005. 540 p.
	
	%\bibitem{Kanungo2000}
	%\textit{T. Kanungo, D. M. Mount et al} An Efficient k-Means Clustering Algorithm: Analysis and Implementation. 2000.
	
	\bibitem{Shiglavsi1997}
	\textit{Д. Л. Данилова, А. А. Жигловский} Главные компоненты временных рядов: метод "Гусеница".~---~Санкт-Петербурскиий университет, 1997.
	

	
\end{thebibliography}