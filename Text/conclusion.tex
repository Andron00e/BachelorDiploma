\newpage

\section{Заключение}
\begin{table}[h!t]
\begin{center}
\caption{Результаты работы алгоритма}
\label{table_2}
\begin{tabular}{|c|c|c|c|c|}
\hline
	Ряд,~$\textbf{x}$ &Длина ряда,~$N$& \# сегментов,~$K$&Длина сегмента,~$T$& Ошибка,~$S$\\
	\hline
	\multicolumn{1}{|l|}{Phys.~Motion~1}
	& 900& 2& 40& 0.06\\
	\hline
	\multicolumn{1}{|l|}{Phys.~Motion~2}
	& 900& 2& 40& 0.03\\
	\hline
	\multicolumn{1}{|l|}{Synthetic~1}
	& 2000& 2& 20& 0.04\\
	\hline
	\multicolumn{1}{|l|}{Synthetic~2}
	& 2000& 3& 20& 0.03\\
\hline

\end{tabular}
\end{center}
\end{table}

В работе рассматривалась задача поиска характерных периодических структур внутри временного ряда. 
Рассматривался метод основаный на локальном снижение размерности фазового пространства. 
Был предложен алгоритм поиска характерных сегментов, который основывается на методе главных компонент для локального снижения размерности. 
Также введена функция расстояния между локальными базисами в каждый момент времени, которые интерпретировались как признаковое описание точки временного ряда.

В ходе эксперимента, на реальных показаниях акселерометра, а также на синтетических данных, было показано, что предложенный метод измерение расстояния между базисами хорошо разделяет точки которые принадлежат различным действиям, что приводит к хорошей кластеризации объектов. 
Результаты работы показаны в таблице~\ref{table_2}.
Также в эксперименте была проведена полная сегментация временных рядов при помощи метода~\cite{motrenko2015} для каждого кластера по отдельности.

Предложенный метод имеет ряд недостатков связанных с большим число ограничений на временной ряд.
Данные ограничения будут ослаблены в последующих работах. Планируется решить задачу нахождения и описания замкнутой фазовой траектории, которая относится к одному квазипериодическому сегменту. 
