\newpage


\section{Постановка задачи кластеризации точек временного ряда}

Задан временной ряд
\begin{equation}
\label{eq:st:1}
\begin{aligned}
\textbf{x} \in \mathbb{R}^{N},
\end{aligned}
\end{equation}
где~$N$ число точек временного ряда. Он состоит из последовательности сегментов:
\begin{equation}
\label{eq:st:2}
\begin{aligned}
\textbf{x} = [\textbf{v}_1, \textbf{v}_2, \cdots, \textbf{v}_M],
\end{aligned}
\end{equation}
где~$\textbf{v}_i$ некоторый сегмент из множества сегментов~$\mathbf{V}$, которые встречаются в данном ряде. 
Причем для всех~$i$ либо~$[\textbf{v}_{i-1},\textbf{v}_{i}]$ либо~$[\textbf{v}_{i},\textbf{v}_{i+1}]$  является цепочкой действий. Пусть множество~$\mathbf{V}$ удовлетворяет следующим свойствам:

\begin{equation}
\label{eq:st:3}
\begin{aligned}
\left|\mathbf{V}\right| = K, \quad \textbf{v} \in \mathbf{V}~\left|\textbf{v}\right| \leq T,
\end{aligned}
\end{equation}
где~$\left|\mathbf{V}\right|$ число различных действий в множестве сегментов $\mathbf{V},$~$\left|\textbf{v}\right|$ длина сегмента, а~$K$ и~$T$ это число различных действий во временном ряде и длина максимального сегмента соответсвенно.

Рассматривается отображение
\begin{equation}
\label{eq:st:4}
\begin{aligned}
a : t \to \mathbb{Y} = \{1,\cdots, K\}, 
\end{aligned}
\end{equation}
где~$t \in \{1,\cdots, N\}$ некоторый момент времени, на котором задан временной ряд.
Требуется, чтобы отображение~$a$ удовлетворяло следующим свойствам:

\begin{equation}
\label{eq:st:5}
\begin{aligned}
\begin{cases}
    a\left(t_1\right) = a\left(t_2\right), &  \text{если в моменты } t_1, t_2 \text{ совершается один тип действий}\\
    a\left(t_1\right) \not= a\left(t_2\right), &  \text{если в моменты } t_1, t_2 \text{ совершаются разные типы действий }
\end{cases}
\end{aligned}
\end{equation}

Пусть задана некоторая асессорская разметка временного ряда:
\begin{equation}
\label{eq:st:6}
\begin{aligned}
\textbf{y} \in \{1,\cdots,K\}^{N}.
\end{aligned}
\end{equation}
Тогда ошибка алгоритма~$a$ на временном ряде~$\textbf{x}$ представляется в следующем виде:
\begin{equation}
\label{eq:st:7}
\begin{aligned}
S = \frac{1}{N}\sum_{t=1}^{N}[y_t = a\left(t\right)],
\end{aligned}
\end{equation}
где~$t$~---~момент времени,~$y_t$ асессорская разметка~$t$-го момента времени для заданого временного ряда.
