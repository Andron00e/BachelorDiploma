\newpage


\begin{abstract}
Данная работа посвящена анализу периодических сигналов во временных рядах с целью распознавания физических действий человека с помощью акселерометра. Предлагается метод кластеризации точек временного ряда для поиска характерных квазипериодических сегментов временного ряда. Временные ряды являются объектами сложной структуры, для которых не задано исходное признаковое описание. В качестве признакового описания точек временного ряда рассматриваются главные компоненты локальной окрестности фазовой траектории вблизи данной точки. Для оценки близости двух точек временного ряда вычисляется расстояние между данными точками в построенном пространстве признаков. При помощи матрицы попарных расстояний между точками временного ряда выполняется кластеризация данных точек. Для анализа качества представленного алгоритма проводятся эксперименты на синтетических данных и данных полученных при помощи мобильного акселерометра. Проводится эксперимент с поиском начала квазипериодических сегментов внутри каждого кластера.


\smallskip
\textbf{Ключевые слова}: временные ряды; кластеризация; сегментация; распознавание физической активности; метод главных компонент.
\end{abstract}




