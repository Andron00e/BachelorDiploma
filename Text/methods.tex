\newpage

\section{Кластеризация точек}
Рассмотрим фазовую траекторию временного ряда~$\textbf{x}$:
\begin{equation}
\label{eq:cl:1}
\begin{aligned}
\mathbf{H} = \{\textbf{h}_t| \textbf{h}_t = [x_{t-T}, x_{t-T+1}, \cdots, x_{t}],~T\leq t\leq N\},
\end{aligned}
\end{equation}
где $\textbf{h}_t$~---~точка фазовой траектории.

Информация об длине максимального сегмента~$T$ внутри временного ряда позволяет разбить фазовую траекторию на сегменты из~$2T$ векторов:
\begin{equation}
\label{eq:cl:2}
\begin{aligned}
\mathbf{S} = \{\textbf{s}_t| \textbf{s}_t = [\textbf{h}_{t-T}, \textbf{h}_{t-T+1}, \cdots, \textbf{h}_{t+T-1}],~T\leq t\leq N-T\},
\end{aligned}
\end{equation}
где $\textbf{s}_t$~---~это сегмент фазовой траектории. Данные сегменты имеют всю локальную информацию об временном ряде, так как содержит всю информацию на периоде до момента времени~$t$ и информацию о периоде после момента времени~$t$.

В качестве признакового описания точки временного ряда $t$ рассматриваются главные компоненты~$\textbf{W}_t$ для~$T\text{-мерных}$ сегментов~$\textbf{s}_t$. Сегмент~$\textbf{s}_t$ проекцируется на подпространство размерности два при помощи метода главных  компонент~$\textbf{z}_t~=~\textbf{W}_t\textbf{s}_t$. Получаем:

%Каждое~$T\text{-мерное}$ подпространство~$\textbf{s}_t$ проектируется на подпространство значительно меньшей размерности при помощи метода главных  компонент~$\textbf{z}_t~=~\textbf{W}_t\textbf{s}_t$. Получим представление базисных векторов~$\textbf{W}_t$, а также собственные числа, которые соответствуют данным базисным векторам каждого подпространства~$\textbf{s}_t$ в~$T\text{-мерном}$ пространстве:
\begin{equation}
\label{eq:cl:3}
\begin{aligned}
\mathbf{W} = \{\textbf{W}_t| \textbf{W}_t = [\lambda^1_t\textbf{w}^1_t, \lambda^2_t\textbf{w}^2_t]\}, \quad \bm{\Lambda} = \{\bm{\lambda}_t| \bm{\lambda}_t=[\lambda^1_t, \lambda^2_t]\},
\end{aligned}
\end{equation}
где~$[\textbf{w}^1_t, \textbf{w}^2_t]$ и~$[\lambda^1_t, \lambda^2_t]$ это базисные векторы и соответствующие им собственные для сегмента фазовой траектории~$\textbf{s}_t$.

Для кластеризации точек временного ряда рассмотрим функцию расстояния между элементами~$\mathbf{W}_{t_1},\mathbf{W}_{t_2}$:
\begin{equation}
\label{eq:cl:4}
\begin{aligned}
\rho\left(\textbf{W}_1, \textbf{W}_2\right) = \max\left(\max_{\textbf{e}_2 \in \textbf{W}_2} d_{1}\left(\textbf{e}_2\right), \max_{\textbf{e}_1 \in \textbf{W}_1} d_{2}\left(\textbf{e}_1\right)\right),
\end{aligned}
\end{equation}
где ~$\textbf{e}_i$ это базисный вектор пространства~$\textbf{W}_i,$ а~$d_i\left(\textbf{e}\right)$ является расстоянием от вектора~$\textbf{e}$ до пространства~$\textbf{W}_i$.

\begin{comment}
\begin{theorem}
Пусть задано множество подпространств~$\mathbb{W}$ пространства~$\mathbb{R}^{n}$. Каждое подпространство которого задается базисом~$\mathbf{W}_i\in \mathbf{W}$, тогда функция расстояния~$\rho\left(\textbf{W}_1, \textbf{W}_2\right)$ является метрикой заданой на множестве базисов~$\mathbf{W}$:
\begin{equation}
\begin{aligned}
\rho\left(\textbf{W}_1, \textbf{W}_2\right) = \max\left(\max_{\textbf{e}_2 \in \textbf{W}_2} d_{1}\left(\textbf{e}_2\right), \max_{\textbf{e}_1 \in \textbf{W}_1} d_{2}\left(\textbf{e}_1\right)\right),
\end{aligned}
\end{equation}
где~$\textbf{e}_i$ это базисный вектор из~$\textbf{W}_i$, a~$d_i\left(\textbf{e}\right)$ является расстоянием от вектора~$\textbf{e}$ до пространства заданого базисом~$\textbf{W}_i$.
\end{theorem}

В силу теоремы~\ref{th:1} функция расстояния~(\ref{eq:cl:4}) является метрикой, доказательство данной теоремы представлено в приложении \ref{ProofTheorem1}. 
\end{comment}

В случае, когда все подпространства~$\textbf{W}_t$ имеют размерность два, расстояние~$\rho\left(\textbf{W}_1, \textbf{W}_2\right)$ имеет следующую интерпретацию:

\begin{equation}
\label{eq:cl:5}
\begin{aligned}
\rho\left(\textbf{W}_1, \textbf{W}_2\right) = \max_{\{\textbf{a},\textbf{b},\textbf{c}\} \subset \textbf{W}_1\cup \textbf{W}_2 } V\left(\textbf{a},\textbf{b},\textbf{c}\right), 
\end{aligned}
\end{equation}
где~$\textbf{W}_1\cup\textbf{W}_2$ это объединение базисных векторов первого и второго пространства,~$V\left(\textbf{a},\textbf{b},\textbf{c}\right)$~---~объем параллелепипеда построенного на векторах~$\textbf{a}, \textbf{b}, \textbf{c}$, которые являются столбцами матрицы~$\textbf{W}_1\cup\textbf{W}_2$.


Рассмотрим расстояние между собственными числами:
\begin{equation}
\label{eq:cl:6}
\begin{aligned}
\rho\left(\bm{\lambda}_1, \bm{\lambda}_2\right) = \sqrt[]{\left(\bm{\lambda}_1 - \bm{\lambda}_2\right)^{\mathsf{T}}\left(\bm{\lambda}_1 - \bm{\lambda}_2\right)}.
\end{aligned}
\end{equation}
%Функция расстояния~$\rho\left(\bm{\lambda}_1, \bm{\lambda}_2\right)$ является метрикой в пространстве~$\mathbb{R}^2$.
%Матрица попарных расстояний между базисными векторами~$\textbf{M}_{\text{c}}$ и матрица попарных расстояний между собственными значениями~$\textbf{M}_{\text{l}}$ для временного ряда~$\textbf{x}$:
%\begin{equation}
%\label{eq:cl:8}
%\begin{aligned}
%\textbf{M}_{\text{c}} = [0, 1]^{N\times N}, \quad \textbf{M}_{\text{l}} = [0, 1]^{N\times N}.
%\end{aligned}
%\end{equation}
Используя выражения~(\ref{eq:cl:5}-\ref{eq:cl:6}) введем расстояние между двумя точками~$t_1, t_2$ временного ряда, а также рассмотрим матрицу попарных расстояний~$\textbf{M}$ между точками данного ряда:
\begin{equation}
\label{eq:cl:9}
\begin{aligned}
\rho\left(t_1, t_2\right) = \rho\left(\textbf{W}_1, \textbf{W}_2\right) + \rho\left(\bm{\lambda}_1, \bm{\lambda}_2\right), \quad \textbf{M} =  \mathbb{R}^{N\times N},
\end{aligned}
\end{equation}
где %~$\rho\left(t_1, t_2\right)$ является метрикой, так как является суммой двух метрик,
матрица~$\textbf{M}$ является матрицей попарных расстояний между всеми парами точек~$t$ временного ряда~$\textbf{x}$.
Используя матрицу попарных расстояний~$\textbf{M}$ выполним кластеризацию моментов времени~$t$ временного ряда~\eqref{eq:st:4}:

%\begin{equation}
%\label{eq:cl:10}
%\begin{aligned}
%a : t \to \{1,\cdots, K\}, 
%\end{aligned}
%\end{equation}
%где~$t$ некоторый момент времени временного ряда~$\textbf{x}$.

