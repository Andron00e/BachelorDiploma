\newpage

\begin{abstract}
Мы предлагаем новую архитектуру и метод объяснимой классификации с использованием концептуальных боттлнек моделей (CBM). В то время как SOTA подходы к задаче классификации изображений работают как черный ящик, растет спрос на модели, которые могли бы предоставлять интерпретируемые результаты. Такие модели часто учатся предсказывать распределение по меткам классов, используя дополнительное описание этих целевых экземпляров, называемое концепциями. Однако существующие методы определения боттлнеков имеют ряд ограничений: их точность ниже, чем у стандартной модели, и CBM требуют дополнительного набора концепций для использования. Мы предоставляем основу для создания концептуальной боттлнек модели на основе предварительно обученного мультимодального энкодера и новых CLIP-подобных архитектур. Представляя новый тип слоев, известный как концептуальные боттлнек слой, мы описываем три метода их обучения: с использованием $\ell_1$-потерь, контрастивных потерь и функции потерь, основанной на распределении Gumbel-Softmax (Sparse-CBM), в то время как конечный слой FC по-прежнему обучается с использованием перекрестной энтропии. Мы демонстрируем значительное повышение точности при использовании разреженных скрытых слоев в CBM на основе CLIP. Это означает, что разреженное представление вектора активации концепций имеет смысл в CBM. Более того, с помощью нашего алгоритма поиска по матрице концепций мы можем улучшить предсказание CLIP в сложных наборах данных без какого-либо дополнительного обучения или файн-тьюнинга. Код доступен по ссылке: \texttt{https://github.com/Andron00e/SparseCBM}.
\end{abstract}




