\documentclass[12pt]{article}
\usepackage[utf8]{inputenc}
\usepackage[english,russian]{babel}

\textheight=24.5cm % высота текста
\textwidth=16cm % ширина текста
\oddsidemargin=0pt % отступ от левого края
\topmargin=-1.5cm % отступ от верхнего края
\parindent=24pt % абзацный отступ
\parskip=0pt % интервал между абзацами
\tolerance=2000 % терпимость к "жидким" строкам
\flushbottom % выравнивание высоты страниц

\begin{document}
\thispagestyle{empty}
\begin{center}
\bigskip

\textbf{Отзыв на бакалаврскую диссертацию студента 4 курса\\
Семёнова Андрея\\
<<Контрастное обучение в задачах компьютерного зрения для повышения интерпретируемости модели>>}
\end{center}

В бакалаврской работе A. Семёнова рассматривается задача об интерпретируемой классификации изображений. В качестве основной модели выбран подход основанный на контрастном предобучении пар изображений и текстов -- CLIP. Данный метод позволяет строить общее признаковое пространство разных доменов данных, чем и полезен в научном сообществе с 2021 года.

В работе было предложено два подхода. Первый основывается на повышении интерпретируемости исходной модели CLIP через использование семантических свойств набора концептов и связь их векторный представлений с векторными представлениями изображений в одном признаковом пространстве. Второй метод -- повышает итоговое качество классификации всей модели, обучая, при этом, внутренние слои так, чтобы они производили разреженные активации. Было показано, что на интерпретируемость предсказаний положительно влияют разреженные выходы именно внутри модели, ранее, разреженность использовали лишь в последнем слое осуществляющем финальную классификацию. В статье Андрей предложил три независимых способа введения разреженности во внутренние слои при обучении нейронных сетей. Стабильность обучения модели обусловлена популярной техникой клиппинга использующегося из-за явлений тяжелого шума в задачах обработки ествественного языка. Все результаты подтвержены вычислительными экспериментами по сравнению с предшествующими статьями.

За время выполнения работы А. Семёнов продемонстрировал способность самостоятельно решать поставленную задачу, а также творчески подходить к поиску ее решения. Следует отметить аккуратное и квалифицированное выполнение численных экспериментов и разработку программной системы, решающей поставленную задачу.

В течение обучения в бакалавриате А. Семёновым были опубликованы работы <<Bregman Proximal Method for Efficient Communications under Similarity>>, <<Sparse Concept Bottleneck Models: Gumbel Tricks in Contrastive Learning>>, <<Gradient Clipping Improves AdaGrad when the Noise Is Heavy-Tailed>>, а также были подготовлены ряд работ, которые готовы к публикации.

Работа является актуальным исследованием, удовлетворяет требованиям, предъявляемым к бакалаврским диссертациям в МФТИ, и заслуживает оценки <<отлично>>, а А. Семёнов ~---~присвоения квалификации бакалавра и рекомендации в магистратуру.


\vspace{3cm}
\begin{flushleft}
Научный руководитель:\\
канд.\,ф.-м.\,н.\\
\end{flushleft}
\begin{flushright}
А.\,Н. Безносиков
\end{flushright}



\end{document}
